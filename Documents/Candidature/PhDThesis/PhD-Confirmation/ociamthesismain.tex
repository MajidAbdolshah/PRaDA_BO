\documentclass[12pt]{ociamthesis}  % default square logo 
%\documentclass[12pt,beltcrest]{ociamthesis} % use old belt crest logo
%\documentclass[12pt,shieldcrest]{ociamthesis} % use older shield crest logo

%load any additional packages
\usepackage{amsmath}
\usepackage{amssymb}
\usepackage{hyperref}
\usepackage{algorithm}% http://ctan.org/pkg/algorithm
\usepackage{rotating}
\usepackage{adjustbox}
\usepackage{dirtytalk}
\usepackage{rotating}
\usepackage{pdflscape}
\usepackage{algpseudocode}% http://ctan.org/pkg/algorithmicx
\usepackage{array,longtable}
\usepackage{subfig}
\usepackage{geometry}
\usepackage{afterpage}
\usepackage{amsmath}
\DeclareMathOperator{\dis}{d}
\renewcommand*{\arraystretch}{1.5}
\newcolumntype{L}[1]{>{\raggedleft\let\newline\\\arraybackslash\hspace{0pt}}m{#1}}
\newcolumntype{C}[1]{>{\centering\let\newline\\\arraybackslash\hspace{0pt}}m{#1}}
\newcolumntype{R}[1]{>{\raggedright\let\newline\\\arraybackslash\hspace{0pt}}m{#1}} 
%input macros (i.e. write your own macros file called mymacros.tex 
%and uncomment the next line)
%\include{mymacros}
\usepackage{xcolor,colortbl}

% A package which allows simple repetition counts, and some useful commands

\usepackage{forloop}
\newcounter{loopcntr}
\newcommand{\rpt}[2][1]{%
  \forloop{loopcntr}{0}{\value{loopcntr}<#1}{#2}%
}
\newcommand{\on}[1][1]{
   \forloop{loopcntr}{0}{\value{loopcntr}<#1}{&\cellcolor{gray}}
}
\newcommand{\onx}[1][1]{
  \forloop{loopcntr}{0}{\value{loopcntr}<#1}{&\cellcolor{orange}}
}
\newcommand{\ony}[1][1]{
  \forloop{loopcntr}{0}{\value{loopcntr}<#1}{&\cellcolor{blue}}
}
\newcommand{\onj}[1][1]{
  \forloop{loopcntr}{0}{\value{loopcntr}<#1}{&\cellcolor{majid}}
}
\newcommand{\ong}[1][1]{
  \forloop{loopcntr}{0}{\value{loopcntr}<#1}{&\cellcolor{green}}
}
\newcommand{\off}[1][1]{
  \forloop{loopcntr}{0}{\value{loopcntr}<#1}{&}
}


\definecolor{orange}{HTML}{BF3929}
\definecolor{blue}{HTML}{45AAF4}
\definecolor{majid}{HTML}{467597}
\definecolor{green}{HTML}{78A73A}


\title{Multi-objective Bayesian \\[1ex]     %your thesis title,
       Optimisation and Applications}   %note \\[1ex] is a line break in the title

\author{Majid Abdolshah}             %your name
\college{School of IT}  %your college

%\renewcommand{\submittedtext}{change the default text here if needed}
\degree{Doctor of Philosophy}     %the degree
\degreedate{Trinity 1998}         %the degree date

%end the preamble and start the document
\begin{document}

%this baselineskip gives sufficient line spacing for an examiner to easily
%markup the thesis with comments
\baselineskip=18pt plus1pt

%set the number of sectioning levels that get number and appear in the contents
\setcounter{secnumdepth}{3}
\setcounter{tocdepth}{3}


\maketitle                  % create a title page from the preamble info0
%\begin{dedication}
This thesis is dedicated to\\
 someone\\
for some special reason\\
\end{dedication}        % include a dedication.tex file
%\begin{acknowledgements}
plenty of waffle, plenty of waffle, plenty of waffle, plenty of waffle,
plenty of waffle, plenty of waffle, plenty of waffle, plenty of waffle.
\end{acknowledgements}   % include an acknowledgements.tex file	
\begin{abstract}
plenty of waffle, plenty of waffle, plenty of waffle, plenty of waffle,
plenty of waffle, plenty of waffle, plenty of waffle, plenty of waffle.
\end{abstract}          % include the abstract

\begin{romanpages}          % start roman page numbering
\tableofcontents            % generate and include a table of contents
\listoffigures              % generate and include a list of figures
\listoftables
\end{romanpages}            % end roman page numbering

%now include the files of latex for each of the chapters etc
\chapter{Introduction}
Bayesian optimisation is considered to be the specific case of model-based optimisation based on the Bayesian formulations. Like most Bayesian approaches, there is a prior distribution over a function, a likelihood function, and a posterior distribution over the unknown functions given data \cite{gelbart2015constrained}. Bayesian optimisation aims to solve a difficult though concise problem:
\begin{equation}
	\operatorname*{min}_{x \in \mathcal{X}} f(x)	
\label{eq:ch1_1}
\end{equation}
where $f(x)$ has the following features \cite{gelbart2015constrained}:
\begin{enumerate}
\item \textbf{Expensive to Evaluate: } Evaluation of $f(x)$ is highly expensive. Such as most of engineering problems like making alloys \cite{vellanki2017process}, designing a pharmaceutical drug, optimising machine learning algorithms \cite{snoek2012practical} and many other engineering problems.
\item \textbf{Black-box: } The exact formulation of $f(x)$ function and/or the derivatives of $f(x)$ is not available. Like the most of the engineering problems that we have already mentioned.
\item \textbf{Noisy Evaluations: } The value of $f(x)$ for a specific $x$ usually corrupted with a noise. So for the same values of $x$, $f(x)$ would result in different values.
\item \textbf{Global and non-convex: } Bayesian optimization of $f(x)$ is a global optimisation problem for a non-convex function in the $\mathcal{X}$ domain.
\end{enumerate}
While traditional numerical methods have proved ineffective for solving some optimisation problems, Bayesian optimisation has proved to be effective in variety of optimisation problems dealing with black-box objective functions taht are expensive to evaluate. There have been number of studies on the use of Bayesian optimisation on hyperparameter tuning in machine learning and big data \cite{joy2016hyperparameter}, expensive multi-objective optimisation for Robotics \cite{tesch2013expensive}, and experimentation optimisation in product design such as short polymer fiber materials \cite{li2017rapid}.
\par
Practical problems are often involved in several non-commensurable objectives. In other words, the real-world problems consist of multiple, conflicting black-box objectives. In Multi-Objective Optimisation (MOO) potential solutions are assessed  by their performance in more than one objective \cite{couckuyt2014fast}. In MOO, based on definition of Pareto optimality, we wish to return a Pareto front that represents the best trade-off  possible considering all criteria \cite{calandra2014pareto}. A solution is Pareto optimal if no objective can be improved on without making at least one other objective worse. More generally, MOO includes $M$ objective functions $f_1, f_2, \ldots, f_M$ which are usually modelled by Gaussian Process (GP) \cite{rasmussen2006gaussian}. Formally:
\begin{equation*}
\begin{aligned}
& \underset{\textbf{x}}{\text{minimize}}
& & \textbf{y}=\textbf{Z(x)}=\{f_1(\textbf{x}),f_2(\textbf{x}), \ldots,f_M(\textbf{x})\} \\
& \text{where:}
& & {\textbf x} \in \mathcal{X} \subseteq \mathbb{R}^d\\
& \text{}
& & {\textbf y} \in \mathcal{Y} \subseteq \mathbb{R}^M\\
\end{aligned}
\end{equation*}
\par
Generally two categories of solutions are available for solving multi-objective problems. In such scenarios it is customary to seek for a set of Pareto optimal outcomes often called the Pareto front \cite{couckuyt2014fast}. The first category uses scalarization for transforming a multi-objective problem into single-objective problem. The single-objective problem is then tackled using different optimisation methods.
To obtain an accurate approximation of the whole Pareto set, sequence of auxiliary single-objective optimization problems should be solved \cite{tan2002evolutionary}. So in this case, such solutions are not suited for the highly expensive and black-box $f(x)$ function. In such studies, there are differnt methods of scalarization in order to solve the multi-objective problems. Tchebycheff Method, $k^{\text{th}}$-Objective Weighted-Constraint, and  Pascoletti and Serafini Scalarization are the three most popular approaches for scalarization among researchers \cite{pardalos2017non}.
\par
The next category of solutions are focused on the multi-objective structure of the problem. These methods do not transform the multi-objective problems into single-objective space. Instead, these methods tend to use more complex acquisition functions based on Expected Hypervolume Improvement (EHI) or Entropy Search (ES) in order to handle the multidimensionality of the problem. So they provide a set of solutions by optimizing all $M$ objective simultaneously. Also there are many well-known evolutionary-based methods in this category such as NSGA-II \cite{deb2002fast} or SPEA2 \cite{zitzler2001spea2}.

\section{Motivation}\label{sec:mot}
Large diversity of optimisation problems take the form described above. Examples include increasing the Pouring Temperature of an alloy to withstand enormously high temperature of molten metals while overall alloy hardness must not increase drastically; or tuning training parameters of an SVM‌ model to maximize the accuracy and minimizing the consumption of resources. In addition to expensive evaluation of objective functions, many optimisation problems deal with similarly expensive black-box constraints. Unknown constraints are part of many black-box multi-objective optimisation problems. For example, when tuning SVM hyperparameters we may want to optimise performance subject to a limit on the number of support vectors (and hence the complexity of evaluating the trained classifier) if the trained machine is to be implemented on limited hardware (such as accessible memory). The goal of optimisation
in such cases is to minimize the number of black-box function evaluations to find the global optimum of the function with respect to constraints. Bayesian optimisation has recently been used successfully in this area \cite{MajidPaper2018}.
Another use of multi-objective with constraints problem arises when there is a priori such as Ranking of the objectives. In many real-world problems, there are more important objectives we would like to have a slight advantage over other objectives. For example, in autonomous flying airplanes, the nature of the problem demands a huge advantage on safety measures of the airplane than the fuel consumption; or in Robotics, one of the current problems researches are working on, is about overheating of motors or power consumption of them. While other objectives such as the accuracy of movements with some constraints are also play a role in this problem. One may like to focus on the safety of motors or the accuracy of the movements. So ranking of the objectives as a priori information could be really useful in many real-world problems.
%
%\section{Summary of contributions}
%The main contribution of our first proposed method is to characterize a general formulation for Multi-objective Bayesian optimisation with unknown constraints based on hypervolume calculation. The other related contributions are:
%\begin{itemize}
%\item Formulation of the expected hypervolume improvement with constraints based on the simple but effective expected improvement acquisition function.
%\item Evaluation of the proposed algorithm based on feasible dominated region on all related benchmark test functions for the first time. We also estimated the volume of the feasible region of the test functions for more accurate evaluation.
%\item Discussion of the issues involved in the method in terms of the efficiency and size of the problem.
%\end{itemize}
%The main contribution of multi-objective Bayesian optimisation with constraints and objective rankings is incorporating the rankings as a feature to multi-objective Bayesian optimisation. The contributions of the proposed model are summarized in the following three aspects:
%\begin{itemize}
%\item We formulated a single-objective constrained Bayesian optimisation problem for mapping a multi-objective Bayesian optimisation with constraints and 
%objective rankings into a more feasible space.
%\item For evaluation of the proposed algorithm, we will introduce new measurements which should contain both diversity/density and accuracy of the obtained Pareto set.
%\item It is the first time that multi-objective Bayesian optimisation has been incorporated in ranking or planning problems.
%\end{itemize}

\chapter{Literature Review}\label{sec:lit}


Bayesian optimisation is a well-known tool for solving a variety of optimisation problems. While traditional numerical methods have proved ineffective for solving some optimisation problems, Bayesian optimisation has proved effective in variety of optimisation problems dealing with blackbox objective functions expensive to evaluate \cite{gelbart2014bayesian}. Bayesian optimization is impacting a wide range of areas, including Robotics \cite{lizotte2007automatic,martinez2007active}, environmental monitoring \cite{marchant2012bayesian}, interactive user interface \cite{brochu2010bayesian}, information extraction \cite{wang2014bayesian}, combinatorial optimisation \cite{wang2013bayesian,hutter2011sequential}, reinforcement learning \cite{brochu2010tutorial}, sensor networks \cite{garnett2010bayesian,srinivas2009gaussian}, and automated machine learning algorithms \cite{thornton2013auto,snoek2012practical,hoffman2014correlation,garnett2013active}.
\par
Fundamentally, Bayesian optimization is a sequential model-based approach for solving optimisation problems. Bayesian optimisation framework has two main stages. The first one is a probabilistic surrogate model, consist of a prior distribution which encodes our beliefs about the nature of the expensive black-box function \cite{shahriari2016taking}. The second stage is constructing a proper acquisition function which can accurately model the behavior of the black-box function. Equipped with these probabilistic models, acquisition functions will be sequentially induced  in order to leverage the uncertainty in the posterior for leading the exploration \cite{shahriari2016taking}. After observing the output value of each selected point in that iteration, the prior understanding of the black-box function and the acquisition function will be updated subsequently. Algorithm \ref{alg:1} illustrates the procedure of Bayesian optimisation.
\par
But as we have mentioned before, many real-world optimisation problems are dealing with unknown constraints. In the next section we are investigating the role of unknown constraints in single-objective Bayesian optimisation.
\begin{algorithm}
  \caption{Bayesian optimisation Algorithm}\label{algo1}
  \begin{algorithmic}[1]
 	  \For{$n= \ 1,2,..., $}
	  \State Optimise acquisition function $\alpha$,
	  $\textbf{x}_{t+1} = \underset{\textbf{x}   \in \mathcal{X}}{\mathrm{argmax}\ \alpha(\textbf{x};\mathcal{D}_n)}$
	  \State Evaluate $\textbf{x}_{t+1}$, obtain ${y}_{t+1}$
	  \State Augment data to the observation set, $\mathcal{D}_{n+1} = \{\mathcal{D}_n,(\textbf{x}_{n+1},{y}_{n+1})\}$
	  \State Update the prior model based on the new observed point.
	  \EndFor
  \end{algorithmic}
  \label{alg:1}
\end{algorithm}

\section{Constrained single-objective Bayesian optimisation}
Consider an SVM model; when tuning SVM hyperparameters, we may want to optimise performance on limited hardware (such as accessible memory). So, in addition to expensive evaluations of the objective function, we may face with similarly expensive evaluations of constraint functions. The only single-objective Bayesian optimisation with constraints method is proposed by \cite{gardner2014bayesian}. In this paper, the authors have extended Bayesian optimization to incorporate inequality constraints. The aim of this method is:
\begin{equation}
\operatorname*{min}_{c(\textbf{x}) \leq \lambda} f(\textbf{x})	
\label{eq:1}
\end{equation}
where $f(\textbf{x})$ and $c(\textbf{x})$ are both expensive and black-box functions. Authors defined a new \textit{Constrained Improvement} in order to form the acquisition function. Constrained improvement is defined as:
\begin{equation}
\mathrm{I}_C(\hat{\textbf{x}}) = \Delta(\hat{\textbf{x}})\ max\{0,f(\textbf{x}^{+})-f(\hat{\textbf{x}})\} = \Delta(\hat{\textbf{x}})\mathrm{I}(\hat{\textbf{x}})
\label{eq:2}
\end{equation}
In equation \ref{eq:2}, $\Delta(\hat{\textbf{x}})$ is defined to be $\Delta(\hat{\textbf{x}}) \in \{0,1\}$ which is a feasibility indicator function that returns $1$ when $c(\hat{\textbf{x}}) \leq \lambda$ and $0$ otherwise. Also $x^+$ denotes a feasible point with lowest function value observed in time $\tau$. Due to dealing with black-box functions for both objectives and constraints, the authors use Bayesian formalism to model each with a GP $\hat{c}(\textbf{x}) \thicksim
 \mathcal{N}(\hat{\mu}_{c}(\textbf{x}),\hat{\Sigma}_c(\textbf{x}))$ and $\hat{f}(\textbf{x}) \thicksim \mathcal{N}(\hat{\mu}_{f}(\textbf{x}),\hat{\Sigma}_f(\textbf{x}))$.
Due to ehe marginal Gaussianity of $\hat{c}(\textbf{x})$, the expected constrained improvement acquisition function is defined as:
\begin{equation}
\begin{split}
\mathrm{EI}_C(\hat{x}) = \mathbb{E}[\mathrm{I}_C(\hat{\textbf{x}})|\hat{\textbf{x}}]
		     		  & \quad = \mathbb{E}[\Delta(\hat{\textbf{x}})\mathrm{I}(\hat{\textbf{x}})|\hat{\textbf{x}}]\\
					  & \quad = \mathbb{E}[\Delta(\hat{\textbf{x}})|\hat{\textbf{x}}]\mathbb{E}[\mathrm{I}(\hat{\textbf{x}})|\hat{\textbf{x}}]\\
   					  & \quad = \mathrm{PF}(\hat{\textbf{x}})\mathbb{E}[\mathrm{I}(\hat{\textbf{x}})
\end{split}
\label{eq:3}
\end{equation}
which the $\mathrm{PF}(\hat{\textbf{x}})$ is defined as a simple univariate Gaussian cumulative distribution function $\mathrm{PF}(\hat{\textbf{x}}) = Pr[{c}(\hat{\textbf{x}}) \leq \lambda] = \int_{-\infty}^{\lambda} p(c(\hat{\textbf{x}})|\hat{\textbf{x}},\tau_c) dc(\hat{\textbf{x}})$. 
Thus the expected constrained improvement acquisition function $\mathrm{EI}_C(\hat{\textbf{x}})$ is the expected improvement of $\hat{\textbf{x}}$ over the best feasible point observed so far. It is also possible to extend the constraint function to the set of independent constraint functions $c(\textbf{x}) = [c_1(\textbf{x}), c_2(\textbf{x}),...,c_k(\textbf{x})]$ \cite{gardner2014bayesian}. In this case, $\mathrm{PF}(\hat{\textbf{x}})$ is multiplication of $k$ constraints as $Pr[c(\textbf{x})_i \leq \lambda_i]_{i=1...k}$. The authors evaluated their proposed model on one simulation function and two real-world problem of Locality Sensitive Hashing and SVM‌ compression. The proposed method proved to be robust in the case of constraints existence.

\section{Multi-objective optimisation and Bayesian optimisation}
There are many studies on the topic of multi-objective optimisation. A quick summary of the most related studies have been illustrated in the Table \ref{tab:1}.

\begin{longtable}{|p{6cm}|p{3cm}|p{2.5cm}|p{2cm}|}
\caption{Summary of related studies in Multi-objective optimisation.}\\
    \hline
    \centering {\bf Study} &\centering {\bf Method} &\centering {\bf Constraints} & {\centering {\bf Ranking}} \\ \hline	

    \centering A Bayesian approach to constrained single- and multi-objective optimization \cite{feliot2017bayesian} & \bf Bayesian optimization  & \centering \checkmark  & -\\\hline    

    \centering Predictive Entropy Search for multi-objective Bayesian optimization with constraints \cite{garrido2016predictive} & \bf Bayesian optimization  & \centering \checkmark  & -\\\hline    

    \centering Predictive Entropy Search for multi-objective Bayesian optimization \cite{hernandez2016predictive} & \bf Bayesian optimisation  & \centering -  & -\\\hline
    \centering Pareto frontier learning with expensive correlated objectives \cite{shah2016pareto} & \bf Bayesian optimisation  & \centering - & - \\\hline    
        \centering Pareto front modeling for sensitivity analysis in multi-objective bayesian optimization \cite{calandra2014pareto} & \bf Bayesian optimisation  & \centering - & - \\\hline
    \centering Active learning of Pareto fronts \cite{campigotto2014active} & Active Learning for Regression Task; Gaussian Processes  &\centering \checkmark & - \\\hline    
\centering A generative kriging surrogate model for constrained and unconstrained multi-objective optimization \cite{hussein2016generative} & Generative surrogate modeling  & \centering \checkmark & -\\\hline    
    \centering Multi-objective reinforcement learning with continuous Pareto frontier approximation \cite{pirotta2015multi} & gradient-based approach & \centering \checkmark & -\\\hline
    \centering Active learning for multi-objective optimization \cite{zuluaga2013active} & \centering Pareto Active Learning (PAL); \centering Gaussian Processes  & \centering -  & {\centering -} \\\hline    
    \centering Faster computation of expected hypervolume improvement \cite{hupkens2014faster} & Expected Hypervolume Improvement \label{tab:1} & \centering -  & -\\\hline
    \centering Multi-objective bandits optimizing the generalized Gini index \cite{busa2017multi} & Generalized Gini Index; Gradient-based algorithm   & \centering - & - \\ \hline

%\centering A novel harmony search algorithm with gaussian mutation for multi-objective optimization \cite{dai2017novel} & Other Text 6  & Other Text 7 & Other Text 6\\\hline   
%\centering Adaptive Weights Generation for Decomposition-Based Multi-Objective Optimization Using Gaussian Process Regression \cite{wu2017adaptive} & 	%Other Text 6  & Other Text 7 & Other Text 6\\\hline    
%    \centering A multi-objective optimization method based on Gaussian process simultaneous modeling for quality control in sheet metal forming %\cite{xia2014multi} & Other Text 6  & Other Text 7 & Other Text 6\\\hline   
%    \centering A Survey on Modeling and Optimizing Multi-Objective Systems \cite{cho2017survey} & Other Text 6  & Other Text 7 & Other Text 6\\\hline

%    \centering The Hypervolume Indicator Revisited On the Design of Pareto-compliant Indicators Via Weighted Integration \cite{zitzler2007hypervolume} & %Other Text 6  & Other Text 7 & Other Text 6\\\hline    
    %\centering An Efficient Batch Expensive Multi-objective Evolutionary Algorithm based on Decomposition  \cite{lin2017efficient} & Other Text 6  & Other Text 7 & Other Text 6\\\hline    
    %\centering A Simple and Fast Hypervolume Indicator-Based Multiobjective Evolutionary Algorithm  \cite{jiang2015simple} & Other Text 6  & Other Text 7 & Other Text 6\\\hline       
    %\centering Multi-Objective Evolutionary Algorithm for a Quick Computation of Pareto-Optimal Solutions  \cite{jiang2015simple} & Other Text 6  & Other Text 7 & Other Text 6\\\hline
    %\centering A fast and elitist multiobjective genetic algorithm: NSGA-II  \cite{deb2002fast} & Other Text 6  & Other Text 7 & Other Text 6\\\hline
    %\centering SPEA2: Improving the strength Pareto evolutionary algorithm  \cite{zitzler2001spea2} & Other Text 6  & Other Text 7 & Other Text 6\\\hline    

\end{longtable}
The authors in \cite{feliot2017bayesian}  proposed a new Bayesian optimization approach to solve multi-objective optimization
problems with non-linear constraints. The constraints are handled by extended domination rule. Also a new expected improvement formulation is proposed. In particular, the new formulation makes it possible to work around the problems in which no feasible solutions are available from the start. 
Sequential Monte Carlo sampling techniques are used in the process of computation and optimization of the new expected improvement criterion. The reason for using Sequential Monte Carlo sampling  is due to no closed-form expression of expected improvement criterion.
The contribution of this article is twofold. The first part of the contribution is about formulation of new sampling criterion that handles multiple objectives and non-linear constraints simultaneously.  The second part of the contribution lies in the numerical methods employed to compute and optimize the sampling criterion \cite{feliot2017bayesian}. 
There is another study about multi-objective optimisation with constraints based on Bayesian optimisation. The authors in \cite{garrido2016predictive} have described an information-based approach which can handle multiple objectives and several constraints. Motivated by the lack of Entropy-based methods, PESMOC is based on the Predictive Entropy Search for multi-objective Bayesian optimization \cite{hernandez2016predictive}. At each iteration, PESMOC evaluates the objective functions and the constraints at an input location that is expected to reduce the entropy of the posterior distribution of the Pareto set the most. The proposed method is most useful useful in practical situations in which the objectives and the constraints are very
expensive to evaluate \cite{garrido2016predictive}.
\par
There are other related studies about multi-objective optimisation. The study conducted in \cite{hernandez2016predictive} described PESMO, a method for multi-objective Bayesian optimization. PESMO evaluates the objective functions at the input location that is most expected to reduce the entropy of posterior estimate of the Pareto set. The structure of acquisition function of PESMO can be understood as a sum of $K$ individual acquisition functions, one per each of the $K$ objectives. This triggers a decoupled evaluation scenario, in which the most promising objective is calculated by maximizing the individual acquisition functions.
The other study in multi-objective Bayesian optimisation is proposed by \cite{shah2016pareto}. The authors focus on modelling correlations amongst
objectives in multi-objective Pareto optimization problems. To overcome the problem of intractable integrals in the proposed method, they have designed a novel approximation which leads to an analytic and differentiable approximation to the expected increase in Pareto hypervolume acquisition function.
Another related study has been conducted in \cite{calandra2014pareto}. The authors showed that by computation of arbitrarily dense and continuous Pareto front, they can approximate the real Pareto front better in presence of measurement noise. These are useful tools to assist the user while making final decision among the Pareto points.
An study based on Active learning is proposed in \cite{campigotto2014active}. Active learning of Pareto fronts framework adopts a different strategy. Pareto-optimal objective vectors are generated by combining the active learning paradigm with the solution of a scalarized optimization problem. The proposed model was iteratively refined until the information gain obtained by the new candidate training examples became negligible.
In another related study, a generative surrogate modeling procedure proposed in \cite{hussein2016generative}. In this work, the authors have proposed a generative surrogate modeling procedure for multi-objective optimization. The main idea is \say{finding a particular Pareto-optimal solution helps in modeling and finding another neighboring Pareto-optimal solution}. 
The authors in \cite{pirotta2015multi} have proposed PMGA, a novel gradient-based approach to learn a continuous approximation of the Pareto frontier in 
multi-objective Markov Decision Problems (MOMDPs). The idea is to define a parametric function that describes a manifold in the policy-parameter space.
The authors have presented different alternatives, discussed about the advantages and disadvantages of the model and shown their properties through an empirical analysis.
The presented model in \cite{zuluaga2013active}, uses a Gaussian processes to predict the objective functions and to guide the sampling process in
order to improve the prediction of the Pareto optimal set. it can be intuitively parameterized to achieve the desired level of accuracy at the lowest possible cost of evaluation. The authors presented an extensive theoretical analysis including bounds for the required number of evaluations to achieve the final accuracy. 
There is an study in which the authors use Gini index in the process of optimisation \cite{busa2017multi}. They introduced a new problem in the context of multi-objective multi-armed bandit (MOMAB). Contrary to most previously proposed approaches that tried to search for the Pareto front, instead they aim for a fair solution. To incorporate the fairness into the formulations, they have used the Generalized Gini Index (GGI), a well-known criterion developed in economics \cite{busa2017multi}.
\section{Multi-objective Bayesian optimisation with constraints}
The goal of optimisation in such cases is to minimize the number of black-box function evaluations to find the global optimum of the function with respect to constraints. Bayesian optimisation with inequality constraints \cite{gardner2014bayesian} and predictive entropy search for Bayesian optimization with unknown constraints \cite{garrido2016predictive} are two major studies investigating the role of inequality black-box expensive constraints in single-objective Bayesian optimisation. There are also two recent studies on multi-objective Bayesian optimisation with constraints.
For example, in \cite{feliot2017bayesian} authors proposed a Bayesian multi-objective optimisation (BMOO) approach to solve the single-objective and multi-objective optimisation with non-linear constraints which is in the same context with the proposed problem in this paper. The method handles
the constraints using an extended Pareto domination rule that takes both objectives and constraints into account. The authors evaluated their method on the benchmark test functions with respect to hypervolume improvement. They proposed approach is inspired from \cite{oyama2007new} which relies on highly complex data models. BMOO uses Sequential Monte-Carlo (SMC) in order to compute the integral over the expected improvement formulation. In \cite{garrido2016predictive}, the authors proposed a method based on predictive entropy search. The authors generated 100 synthetic optimization
problems obtained by sampling the objectives and the constraints from their respective GP prior and they did not use benchmark test functions to evaluate their method. This method is generally categorized as information-based methods while our proposed problem is based on hypervolume improvement approaches

\section{Evolutionary algorithms and Multi-objective problems with constraints}
Generally evolutionary algorithms and Surrogate-assisted evolutionary computation are not designed to work on limited budget of evaluation. But we are covering the overall review of the most related ones. Table \ref{tab:2} illustrates the related studies about such evolutionary methods.
\begin{longtable}{|p{6cm}|p{3cm}|p{2.5cm}|p{2cm}|}
	\caption{Summary of related studies in Evolutionary Multi-objective optimisation.}\\
    \hline
    \centering {\bf Study} &\centering {\bf Method} &\centering {\bf Constraints} & {\centering {\bf Ranking}} \\ \hline	
    \centering A surrogate-assisted evolution strategy for constrained multi-objective optimization \cite{datta2016surrogate} & Surrogate-assisted evolution  & \centering \checkmark &  - \\\hline  
    \centering A simple and fast hypervolume indicator-based multi-objective evolutionary algorithm  \cite{jiang2015simple} & Evolutionary Algorithm; Hypervolume  & \centering - & - \\\hline       
    \centering Multi-objective evolutionary algorithm for a quick computation of Pareto-optimal solutions  \cite{jiang2015simple} & Evolutionary Algorithm  & \centering \checkmark  & -\\\hline
    
    \centering A fast and elitist multiobjective genetic algorithm: NSGA-II  \cite{deb2002fast} & Evolutionary Algorithm  & \centering \checkmark & -\\\hline
    \centering SPEA2: Improving the strength Pareto evolutionary algorithm\label{tab:2}  \cite{zitzler2001spea2} & Evolutionary Algorithm  & \centering \checkmark & -\\\hline    
\end{longtable}
A surrogate-assisted evolution strategy is proposed in \cite{datta2016surrogate}. They have developed a surrogate-assisted multi-objective evolution
strategy (SMES) for computationally expensive constrained multiobjective optimization. 
The main limitation of the present work is that, as with most evolutionary algorithms, there is no theoretical guarantee that the proposed surrogate-assisted ES will converge to the Pareto front. Moreover, the proposed approximations could be inaccurate when the computational budget is severely
limited due to the computational expense of the simulations.
The authors in \cite{jiang2015simple} proposed a way for finding high quality of solutions in indicator-based multiobjective evolutionary algorithms (MOEAs). Hypervolume is a critical performance factor playing a role in solution selection. In this paper, a simple and fast hypervolume indicator-based MOEA (FV-MOEA) is proposed to quickly update exact HV contributions for different solutions. The core idea of FV-MOEA is that the HV
contribution of a solution is only associated with parts of the solutions rather than the full solution set.
The authors in \cite{jiang2015simple} introduced a new method called $\epsilon$-MOEA. They have proven that $\epsilon$-MOEA has two main advantages: 1. It has helped in reducing the cardinality of Pareto-optimal region and and 2. It has also ensured that no two obtained solutions are within an $\epsilon_i$ from each other in the $i-th$ objective.
Also, NSGA-II \cite{deb2002fast} is one of the most famous evolutionary multi-objective algorithm. The authors have proposed a computationally fast and elitist MOEA based on a nondominated sorting approach. The authors believe the proposed method has less computational complexity of nondominated sorting, more elite, and lack of need for specifying the sharing parameter.
The last study is about SPEA2 \cite{zitzler2001spea2}. The authors name many advantages over the similar strategies:
\begin{itemize}
\item SPEA2 performs better than SPEA on all problem sets.
\item PESA has fastest convergence power, probably due to its higher elitism intensity, but has difficulties on some problems regarding the boundary solutions because it does not always keep the boundary solutions.
\item In higher dimensional objective spaces, SPEA2 seems to have advantages over NSGA-II.
\end{itemize}

\section{Multi-objective problems with constraints and objective rankings}
As we have already explained in section \ref{sec:mot}, there are many real-world problems, requiring handling the importance of the objectives. In other words, objective ranking. This information could either be used as priori or as an output of the algorithm. There are not any specific studies on this problem. However, there is an study \cite{kukkonen2007ranking} about the ranking of the solutions based on the objectives. 
The authors studied on an alternative dominance relation to Pareto-dominance. Based on each separate objective, ranking set of solutions will be calculated and an aggregation function is used to calculate a scalar fitness value for each solution. The authors called this relation as \textit{ranking-dominance} and it can be used to sort a set of solutions even for many objectives when Pareto-dominance relation is not able to distinguish solutions from one another. 
\par
Later in this proposal, we will explain our presented idea based on Bayesian optimisation for this particular problem and will present the corresponding results.
\chapter{Background}
This chapter describes the background and related work on which this proposal builds.
We will start from Gaussian process as a power tool for placing the prior belief and compute the uncertainty over the function space. Then we will the  Bayesian optimisation and the acquisition functions. Finally we will explain about multi-objective Bayesian optimisation and Exptected Hypervolume Improvement.

\section{Gaussian Processes}
A Gaussian process is a collection of random variables $\{f(\textbf{x}) : \textbf{x} \in \mathcal{X}\}$ in which any finite collection of random variables has a multivariate
Gaussian distribution \cite{rasmussen2006gaussian}. GPs are determined by using a mean function $\mu(\textbf{x}): \mathcal{X} \rightarrow \mathbb{R}$ and 
a covariance function $k(\textbf{x}): \mathcal{X} \times \mathcal{X} \rightarrow \mathbb{R}^+$:
\begin{equation}
f(\textbf{x}) \sim \mathcal{GP}(\mu(\textbf{x}),k(\textbf{x},\textbf{x}^{\prime})).
\end{equation}
The particular choice of covariance function determines the properties of sample functions drawn from the GP prior.
Popular kernels for covariance function are Squared Exponential, Matern, Periodic, and Linear \cite{wilson2013gaussian}. 
Without loss of generality, the prior mean function generally assumed to be zero in Gaussian process.
Let us assume that we already made $t$ observations on the points $\textbf{x}_{1..t} = \{\textbf{x}_1,...,\textbf{x}_t\}$. The obtained evaluated results of these $t$ observations are $f(\textbf{x}_{1..t}) = \{f(\textbf{x}_1),...,f(\textbf{x}_t)\}$. Based on the defined properties of Gaussian process, the function values  $f(\textbf{x}_{1..t})$ jointly follow a multivariate Gaussian distribution $f(\textbf{x}_{1..t}) \sim \mathcal{N}(0,K)$ \cite{li2017rapid}. Where $K$ is a positive definite kernel matrix:
\[
K = \begin{bmatrix} 
    k(\textbf{x}_1,\textbf{x}_1) & \dots & k(\textbf{x}_1,\textbf{x}_t)\\
    \vdots & \ddots  \\
    k(\textbf{x}_t,\textbf{x}_1) & \dots  & k(\textbf{x}_t,\textbf{x}_t)
    \end{bmatrix}.
\]
\subsection{Computing the Posterior}
The posterior can be similarly derived the way how the update equations for the Kalman filter was derived. First we need to find the joint distribution of $[f(\textbf{x}_{t+1}),f(\textbf{x}_1),\\ f(\textbf{x}_2),...,f(\textbf{x}_t)]$.
Considering $t$  observations of the objective function, for a new point $\textbf{x}_{t+1}$, $P(f(\textbf{x}_{t+1})|\textbf{x}_{1..t},f(\textbf{x}_{1..t})) = \mathcal{N}(\mu_t(\textbf{x}_{t+1}),\sigma^2_t(\textbf{x}_{t+1}))$:
\begin{align}
   \begin{bmatrix} 
    f(\textbf{x}_{t+1})\\
    f(\textbf{x}_{1})\\
    f(\textbf{x}_{2})\\
    \vdots\\
    f(\textbf{x}_{t})
    \end{bmatrix} \sim \mathcal{N} \Bigg(    
   \begin{bmatrix} 
    0\\
    0\\
    \vdots\\
    0
 	\end{bmatrix},\begin{bmatrix} 
    k(\textbf{x}_{t+1},\textbf{x}_{t+1}) & k(\textbf{x}_{t+1},\textbf{x})^T\\
    k(\textbf{x}_{t+1},\textbf{x}) & K_{\textbf{xx}}
    \end{bmatrix}\Bigg)
\end{align}
Where posterior mean and variance are calculated as:
\begin{equation}
\mu_t(\textbf{x}_{t+1}) = k^T  K^{-1}  f(\textbf{x}_{1..t}), 
\label{eq:1}
\end{equation}
\begin{equation}
\sigma^2_t(\textbf{x}_{t+1}) = k(\textbf{x}_{t+1},\textbf{x}_{t+1}) - k^T K^{-1} k.
\label{eq:2}
\end{equation}
$k$ in equation \ref{eq:1} and \ref{eq:2}, is a kernel vector such that $k = [k(\textbf{x}_{t+1},\textbf{x}_1),...,k(\textbf{x}_{t+1},\textbf{x}_t)]^T$.
After modelling the observations with GP, it is required to find the best possible point for next iteration of optimisation ($\textbf{x}_{t+1}$).

\subsection{A few basic kernels}


\include{chapter4}
\include{chapter5}
\include{chapter6}
\include{chapter7}
%\chapter{Sample Title}

Lorem ipsum dolor sit amet, consectetur adipiscing elit, sed do eiusmod tempor incididunt ut labore et dolore magna aliqua. Ut enim ad minim veniam, quis nostrud exercitation ullamco laboris nisi ut aliquip ex ea commodo consequat. Duis aute irure dolor in reprehenderit in voluptate velit esse cillum dolore eu fugiat nulla pariatur. Excepteur sint occaecat cupidatat non proident, sunt in culpa qui officia deserunt mollit anim id est laborum.

%now enable appendix numbering format and include any appendices
\appendix
%\chapter{Sample Title}

Lorem ipsum dolor sit amet, consectetur adipiscing elit, sed do eiusmod tempor incididunt ut labore et dolore magna aliqua. Ut enim ad minim veniam, quis nostrud exercitation ullamco laboris nisi ut aliquip ex ea commodo consequat. Duis aute irure dolor in reprehenderit in voluptate velit esse cillum dolore eu fugiat nulla pariatur. Excepteur sint occaecat cupidatat non proident, sunt in culpa qui officia deserunt mollit anim id est laborum.
%\chapter{Sample Title}

Lorem ipsum dolor sit amet, consectetur adipiscing elit, sed do eiusmod tempor incididunt ut labore et dolore magna aliqua. Ut enim ad minim veniam, quis nostrud exercitation ullamco laboris nisi ut aliquip ex ea commodo consequat. Duis aute irure dolor in reprehenderit in voluptate velit esse cillum dolore eu fugiat nulla pariatur. Excepteur sint occaecat cupidatat non proident, sunt in culpa qui officia deserunt mollit anim id est laborum.

%next line adds the Bibliography to the contents page
\addcontentsline{toc}{chapter}{Bibliography}
%uncomment next line to change bibliography name to references
%\renewcommand{\bibname}{References}
\bibliography{refs}        %use a bibtex bibliography file refs.bib
\bibliographystyle{unsrt}
%\bibliographystyle{plain}  %use the plain bibliography style

\end{document}

